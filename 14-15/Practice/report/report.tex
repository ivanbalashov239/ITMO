\documentclass[a4paper,12pt]{article}
\usepackage[left=2cm,right=2cm,top=2cm,bottom=2cm,bindingoffset=0cm]{geometry}
\usepackage{amsmath}
\usepackage{diagbox}
\usepackage[T1, T2A]{fontenc}	
\usepackage[utf8]{inputenc}	
\usepackage[english, russian]{babel}
\usepackage{amssymb}

\author{Плотников Антон, г. 3743}
\title{Отчет.}
\date{\today}

\begin{document}

\maketitle

\section*{Изученные материалы}
Изучены основы лямбда исчисления и функционального программирования. В частности стратегии редукции, теорема Чёрча-Россера, алгоритмы унификации и Хиндли-Милнера. А так же общие принципы работы с функциональными языками программирования.

\section*{Результат}
В результате реализована небольшая библеокека на языке Haskell для работы с просто типизированными лямбда термами(поиск главной пары, вывод свободных переменных, алгоритм поиска унификатора, различные подстановки). 
\end{document}
